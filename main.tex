\documentclass[12pt]{article}
\usepackage[utf8]{inputenc}
\usepackage[T1]{fontenc}
\usepackage[italian]{babel}

\usepackage{listings}
\usepackage{verbatim}

\usepackage{geometry}
\geometry{ a4paper, left=2cm, right=2cm}

\usepackage{enumitem}
\usepackage{amsmath}
\usepackage{amssymb}

\begin{document}
    \title{Soluzioni quarto appello di calcolo numerico \\ 11 settembre 2019}
    \author{Sara Righetto [AkaiSara]}

    \maketitle

    \section{Esercizio 1}
    \begin{enumerate}
        \item $\alpha _1 \in [-1.25, -0.75]$ , $\alpha _2 \in [-2 , -1.5]$
        \item $f'(x) = 2x +2 - \frac{1}{x}$
        \item \verbatiminput{ris.txt}
        \item $\epsilon _R = \frac{\vert \alpha - x_3 \vert} {\vert \alpha \vert} = \frac{\vert -1 - (-0.999992558) \vert} {\vert -1 \vert} \simeq 0.000007442 \simeq 0.74 \cdot 10^{-5}$ \newline 
        [Nda: la prof. non aveva specificato un certo $\alpha$ per cui calcolare l'errore quindi durante la correzione ha deciso di usare come $\alpha$ la soluzione determinabile analiticamente]
        \item  $x = \sqrt{-2x -1 + \log(-x)}$ , $x = \frac{-x^2 -1 +\log(-x)}{2}$
        \item $p_3 = \frac{\log( \vert x_3 - x_2 \vert / \vert x_2 - x_1 \vert )}{\log(\vert x_2 - x_1 \vert / \vert x_1 - x_0 \vert)} \simeq \frac{-2.8446993250487926}{-1.4268593725791503} \simeq 1.9936788303859232$
    \end{enumerate}

    \section{Esercizio 2}
    \begin{enumerate}
        \item 
            \( U = \begin{pmatrix}
                2 & 1 & 0 \\
                0 & -\frac{1}{2} & 1 \\
                0 & 0 & 2
                \end{pmatrix}
            \) , 
            \( L = \begin{pmatrix}
                1 & 0 & 0 \\
                2 & 1 & 0 \\
                \frac{1}{2} & -1 & 1
                \end{pmatrix}
            \)
        \item Verificato.
        \item $\det(A) = \det(L) \cdot \det(U) = 1 \cdot 2 \cdot -\frac{1}{2} \cdot 2 = -2$ quindi, $\det(A^2) = (-2)^2 = 4$
        \item \( x = \begin{pmatrix}
            1 & 1 & 1
            \end{pmatrix}
        \)$^T$
        \item Per rispondere alla domanda suggerisco le pagine del libro di teoria p.137 - 138 e gli esempi 4.7 (Gauss con pivoting) - 4.3 (Gauss senza pivoting) 
    \end{enumerate}

    \section{Esercizio 3}
    \begin{enumerate}
        \item $m_0 = 2, h_0 = \frac{1}{4}$, $T(h_0) \simeq 0.558\overline{3}$
        \item $m_1 = 4, h_1 = \frac{1}{8}$, $T(h_1) \simeq 0.551605339$
        \item 0.549306144 [Nda: la prof rende già noto sulla consegna la formula per l'integrazione, quindi basta sostituire i valori di $a$ e $b$ dell'intervallo]
        \item $\epsilon _A = \vert \alpha - T(h_0) \vert = \vert -0.009027189 \vert \simeq 0.9 \cdot 10^{-2}$ \newline
         $\epsilon _A = \vert \alpha - T(h_1) \vert = \vert -0.002299195 \vert \simeq 0.23 \cdot 10^{-2}$
    \end{enumerate}


\end{document}